%!TEX TS-program = xelatex
%!TEX encoding = UTF-8 Unicode
% Awesome CV LaTeX Template for CV/Resume
%
% This template has been downloaded from:
% https://github.com/posquit0/Awesome-CV
%
% Author:
% Claud D. Park <posquit0.bj@gmail.com>
% http://www.posquit0.com
%
%
% Adapted to be an Rmarkdown template by Mitchell O'Hara-Wild
% 23 November 2018
%
% Template license:
% CC BY-SA 4.0 (https://creativecommons.org/licenses/by-sa/4.0/)
%
%-------------------------------------------------------------------------------
% CONFIGURATIONS
%-------------------------------------------------------------------------------
% A4 paper size by default, use 'letterpaper' for US letter
\documentclass[11pt, a4paper]{awesome-cv}

% Configure page margins with geometry
\geometry{left=1.4cm, top=.8cm, right=1.4cm, bottom=1.8cm, footskip=.5cm}

% Specify the location of the included fonts
\fontdir[fonts/]

% Color for highlights
% Awesome Colors: awesome-emerald, awesome-skyblue, awesome-red, awesome-pink, awesome-orange
%                 awesome-nephritis, awesome-concrete, awesome-darknight

\definecolor{awesome}{HTML}{203C60}

% Colors for text
% Uncomment if you would like to specify your own color
% \definecolor{darktext}{HTML}{414141}
% \definecolor{text}{HTML}{333333}
% \definecolor{graytext}{HTML}{5D5D5D}
% \definecolor{lighttext}{HTML}{999999}

% Set false if you don't want to highlight section with awesome color
\setbool{acvSectionColorHighlight}{true}

% If you would like to change the social information separator from a pipe (|) to something else
\renewcommand{\acvHeaderSocialSep}{\quad\textbar\quad}

\def\endfirstpage{\newpage}

%-------------------------------------------------------------------------------
%	PERSONAL INFORMATION
%	Comment any of the lines below if they are not required
%-------------------------------------------------------------------------------
% Available options: circle|rectangle,edge/noedge,left/right

\name{José Daniel Conejeros}{}

\position{\href{http://sociologia.uc.cl/postgrado/magister-en-sociologia/}{Magíster(c) en Sociología}, Pontificia Universidad Católica de Chile\\
Licenciado en Ciencias Sociales, Pontificia Universidad Católica de Chile}
\address{Santiago, Chile. Abril 2020}

\mobile{+56 9 64924057}
\email{\href{mailto:jdconejeros@uc.cl}{\nolinkurl{jdconejeros@uc.cl}}}
\github{JDConejeros}
\linkedin{joseconejerosp}

% \gitlab{gitlab-id}
% \stackoverflow{SO-id}{SO-name}
% \skype{skype-id}
% \reddit{reddit-id}

\quote{\textbf{Temas de interés}: pobreza, desigualdad, justicia social, ciencia de datos y metodologías cuantitativas.\\
Autodidacta en la programación para el procesamiento, análisis y visualización de datos sociales.}

\usepackage{booktabs}

% Templates for detailed entries
% Arguments: what when with where why
\usepackage{etoolbox}
\def\detaileditem#1#2#3#4#5{%
\cventry{#1}{#3}{#4}{#2}{\ifx#5\empty\else{\begin{cvitems}#5\end{cvitems}}\fi}\ifx#5\empty{\vspace{-4.0mm}}\else\fi}
\def\detailedsection#1{\begin{cventries}#1\end{cventries}}

% Templates for brief entries
% Arguments: what when with
\def\briefitem#1#2#3{\cvhonor{}{#1}{#3}{#2}}
\def\briefsection#1{\begin{cvhonors}#1\end{cvhonors}}

\providecommand{\tightlist}{%
	\setlength{\itemsep}{0pt}\setlength{\parskip}{0pt}}

%------------------------------------------------------------------------------


\geometry{left=1.4cm, top=1.5cm, right=1.4cm, bottom=1.8cm, footskip=.5cm}
\setbool{acvSectionColorHighlight}{false}
\definecolor{darktext}{HTML}{182D47}
\definecolor{text}{HTML}{595959}
\definecolor{graytext}{HTML}{101E2F}
\definecolor{lighttext}{HTML}{808080}
\newcommand{\bcenter}{\begin{center}}
\newcommand{\ecenter}{\end{center}}


\begin{document}

% Print the header with above personal informations
% Give optional argument to change alignment(C: center, L: left, R: right)
\makecvheader

% Print the footer with 3 arguments(<left>, <center>, <right>)
% Leave any of these blank if they are not needed
% 2019-02-14 Chris Umphlett - add flexibility to the document name in footer, rather than have it be static Curriculum Vitae
\makecvfooter
  {Abril 2020}
    {José Daniel Conejeros~~~·~~~Curriculum Vitae}
  {\thepage}


%-------------------------------------------------------------------------------
%	CV/RESUME CONTENT
%	Each section is imported separately, open each file in turn to modify content
%------------------------------------------------------------------------------



\hypertarget{educaciuxf3n}{%
\section{Educación}\label{educaciuxf3n}}

\detailedsection{\detaileditem{Magíster© en Sociología}{Marzo 2018--Junio 2020}{Pontificia Universidad Católica de Chile}{Santiago, Chile}{\item{Tesista proyecto Fondecyt 1160921: Economía Moral de la Meritocracia y Preferencias Redistributivas.}}\detaileditem{Sociólogo©}{Marzo 2016--Junio 2020}{}{}{\item{Título profesional obtenido con el grado de Magíster}\item{Articulación de estudios programa College UC}}\detaileditem{Licenciado en Ciencias Sociales}{Marzo 2013--Junio 2017}{}{}{\item{Major: Sociología - Minor: Políticas Públicas}\item{Programa College UC}}}

\hypertarget{experiencia-en-investigaciuxf3n}{%
\section{Experiencia en investigación}\label{experiencia-en-investigaciuxf3n}}

\detailedsection{\detaileditem{FONDECYT 1190436 (ref. Oscar Macclure oscar.macclure@gmail.com)}{Septiembre 2019--Diciembre 2020}{Asistente de Investigación}{Santiago, Chile}{\item{Asistente técnico Fondecyt Regular: La posición de uno mismo en la sociedad (lo justo y lo injusto en el espejo social)}\item{Análisis cualitativo de los productos para el año 1 (grupos focales)}\item{Diseño de la etapa cuantitativa del estudio: instrumentos de medición y diseño experimental}\item{Apoyo operativo y reportes de investigación}}\detaileditem{FONDECYT 1181239 (ref. Daniel Miranda damiranda@gmail.com)}{Agosto 2019--Marzo 2020}{}{}{\item{Asistente técnico Fondecyt Regular: Socialización política y educación para la ciudadanía: el rol de la familia y la escuela}\item{Procesamiento, análisis y reporte: Estudio Internacional de Educación Cívica y Formación Ciudadana  (ICCS}\item{Procesamiento, análisis y reporte: Primer Estudio Nacional de Formación Ciudadana (Agencia de calidad de la educación}}\detaileditem{FONDECYT 1160921 (ref. Juan Carlos Castillo juancastillov@uchile.cl)}{Noviembre 2018--Febrero 2019}{Apoyo Técnico}{Santiago, Chile}{\item{Apoyo técnico Fondecyt Regular: Economía moral de la meritocracia y preferencias redistributivas}\item{Construcción de una base de datos con la información de la encuesta Latinobarómetro para el año 2008}\item{Construcción de indicadores y modelos estadísticos en R}}}

\hypertarget{experiencia-profesional}{%
\section{Experiencia profesional}\label{experiencia-profesional}}

\detailedsection{\detaileditem{DataVoz}{Marzo 2020--Abril 2020}{Reporte de datos}{Santiago, Chile}{\item{Diseño y reporte de resultados cuantitativos en Rmarkdown}}\detaileditem{Colaborativa}{Marzo 2020--Abril 2020}{Consultoría Cualitativa}{Santiago, Chile}{\item{Propuesta de diseño cualitativa para licitación pública}}\detaileditem{Vínculo y Desarrollo Profesional College UC (ref. Catherine Calderon, cavcalderon@uc.cl)}{Enero 2019--Agosto 2019}{Staff profesional}{Santiago, Chile}{\item{Análisis, gestión y reporte de datos sobre egresados para el programa College UC}\item{Apoyo profesional en actividades del área}}\detaileditem{Vínculo y Desarrollo Profesional College UC (ref. Catherine Calderon, cavcalderon@uc.cl)}{Marzo 2018--Agosto 2018}{Consultoría Cuantitativa: Caracterización de egresados}{Santiago, Chile}{\item{Caracterización y Seguimiento de Licenciados College UC 2018}\item{Diseño y aplicación de un instrumento on-line}\item{Procesamiento, análisis y presentación de resultados}}\detaileditem{Consejo Nacional de Innovación para el Desarrollo (CNID)}{Febrero 2018--Marzo 2018}{Consultoría Cualitativa: Análisis Lingüístico}{Santiago, Chile}{\item{Proyecto Anticipando Horizontes para Chile: Explorando áreas emergentes en Ciencias y Humanidades}\item{Codificación y análisis lingüístico en mesas de trabajo sobre Ciencia y Humanidades}}\detaileditem{Estudio Racimo}{Febrero 2018--Marzo 2018}{Consultoría Cualitativa: Gestión de cambio}{Santiago, Chile}{\item{Consultoría freelance Proyecto Codelco Tech incorporación tecnología LHD}\item{Análisis de fuentes primarias y material cualitativo}\item{Diseño estratégico para la gestión de cambio en la incorporación de tecnología LHD en el proceso minero}}}

\hypertarget{experiencia-en-docencia}{%
\section{Experiencia en Docencia}\label{experiencia-en-docencia}}

\footnotesize{
Ayudante y colaborador de docencia a nivel universitario en cursos a nivel teórico y metodológico.   
}

\detailedsection{\detaileditem{Profesor auxiliar  Métodos Cuantitativos para Políticas Públicas I (GOB3002)}{2020-S1}{Escuela de Gobierno PUC  (POSTGRADO)}{Santiago, Chile}{\item{Análisis de datos sociales en R}\item{Referencia: Eduardo Undurraga - eundurra@uc.cl}}}

\detailedsection{\detaileditem{Ayudante: Análisis de datos I (SOL106)}{2020-S1}{Instituto de Sociología PUC (PREGRADO)}{Santiago, Chile}{\item{Referencia: Pamela Ayala - pbayala@uc.cl}}\detaileditem{Ayudante: Análisis de datos III (SOL209)}{2019-S2}{}{}{\item{Referencia: Luis Maldonado - lmaldona@uc.cl}}\detaileditem{Ayudante: Metodología de la Investigación Social (SOL126)}{2019-S1}{}{}{\item{Referencia: Consuelo Cheix - mccheix@uc.cl}}\detaileditem{Ayudante: Evaluación de Proyectos Sociales (SOL190)}{2019-S1}{}{}{\item{Referencias: Andrea López - azlopez@uc.cl}}\detaileditem{Ayudante: Sociología Económica (SOL116)}{2018-S1}{}{}{\item{Referencias: Jorge Atria -  jorgeatria@gmail.com}}\detaileditem{Ayudante: Paradigmas Sociológicos (SOL100).}{2016-S2}{}{}{\item{Referencia: Eduardo Galaz - eigalaz@uc.cl}}\detaileditem{}{}{}{}{\empty}\detaileditem{Ayudante: Sociología Clásica}{2019-S1}{Universidad Finis Terrae. Facultad de Educación, Psicología y Familia  (PREGRADO)}{Santiago, Chile}{\item{Referencia: Camila Moyano -  camila.moyano@gmail.com}}\detaileditem{Ayudante: Sociología Moderna}{2018-S2}{}{}{\item{Referencia: Camila Moyano -  camila.moyano@gmail.com}}}

\hypertarget{cursos}{%
\section{Cursos}\label{cursos}}

\detailedsection{\detaileditem{Análisis de Datos Longitudinales I}{2019 - 2020}{Instituo de Sociología PUC}{Santiago, Chile}{\item{Curso de 33 horas para analisis de encuestas longitudinales}\item{Identificación de los principales esquemas de datos longitudinales}\item{Comprender estrategias y metodologías de análisis para datos longitudinales}}\detaileditem{Manejo Eficiente de Datos con STATA}{2019}{}{}{\item{Curso de 25 horas orientado a usuarios de nivel intermedio}\item{Empleo y uso eficiente de funciones}\item{Automarización de análisis para datos transversales}}\detaileditem{Escuela de Métodos Experimentales. Universidad de Santiago - Oxford (Nuffield College)}{2019}{Centre for Experimental Social Sciences (CESS).}{Santiago, Chile}{\item{Curso de 2 semanas introductorio al diseño, implementación y análisis de experimentos en ciencias sociales}\item{Conceptos básicos de inferencia causal}\item{Diseño experimental: laboratorio, campo y web}\item{Programación de base en Python}}}

\break

\hypertarget{habilidades}{%
\section{Habilidades}\label{habilidades}}

\detailedsection{\detaileditem{Análisis}{}{}{}{\item{Modelos estadisticos para variables continuas y categóricas - Modelos econométricos para datos longitudinales - Modelos de ecuaciones estructurales (SEM) - Diseños experimentales e inferencia causal - Investigación reproducible}}\detaileditem{Programación}{}{}{}{\item{R (Intermedio) - STATA (Avanzado) - IBM SPSS (Intermedio) - Python (Inicial)}}\detaileditem{Librerías R (Selección)}{}{}{}{\item{tidyverse - dplyr - car - lme4 - lavaan - ggplot2 - KableExtra}}\detaileditem{Reporte}{}{}{}{\item{Latex - Markdown - Zotero - Tableau}}\detaileditem{Herramientas}{}{}{}{\item{Git - Github - Word, Power Point, Excel (Intermedio - Avanzado)}}\detaileditem{Diseño de encuestas}{}{}{}{\item{QuestionPro - SurveyMonkey - Qualtrics (Inicial)}}\detaileditem{Análisis Cualitativo}{}{}{}{\item{Atlas.ti (Intermedio)}}\detaileditem{Idiomas}{}{}{}{\item{Español (Nativo) - Inglés (Intermedio, PUC)}}}

\hypertarget{presentaciones}{%
\section{Presentaciones}\label{presentaciones}}

\detailedsection{\detaileditem{Universidad de San Marcos}{2019}{Congreso Latinoamericano de Sociología}{Lima, Perú}{\item{Discriminación étnica en la participación de cargos políticos para pueblos originarios: Resultados de un experimento de lista}\item{Educación tóxica: un acercamiento multinivel a una problemática ecológica. Exposición a SO2 y rendimiento académico}\item{Educación, valores y percepciones sobre el mérito: una mirada al caso chileno}}\detaileditem{Universidad de O`Higgins}{2019}{1º Encuentro Nacional de Métodos y Técnicas de investigación Social}{Rancagua, Chile}{\item{Deseabilidad social y sesgo de sensibilidad en discriminación étnica}}\detaileditem{Universidad Arturo Prat}{2018}{10º Congreso de Sociología}{Iquique, Chile}{\item{Desarrollo, educación y cuestiones postmateriales: niveles de aceptación a la homosexualidad en 39 países}}}

\hypertarget{becas-y-premios}{%
\section{Becas y Premios}\label{becas-y-premios}}

\detailedsection{\detaileditem{Beca para estudios de Magíster}{2019-2020}{Centro de Estudios y la Cohesión Social (COES)}{Santiago, Chile}{\empty}\detaileditem{Beca para estudios de Magíster}{}{Fundación Volcán Calbuco}{}{\empty}\detaileditem{Beca para estudios de Magíster}{}{Beca San Andrés, College UC}{}{\empty}\detaileditem{Beca para estudios Técnicos}{2009 -2011}{Beca Fundación Sumate, Hogar de Cristo}{Santiago, Chile}{\empty}}

\hypertarget{otros-antecedentes}{%
\section{Otros antecedentes}\label{otros-antecedentes}}

\detailedsection{\detaileditem{Representación Estudiantil}{2017--2017}{Programa College UC}{Santiago, Chile}{\item{Consejero Académico}}\detaileditem{Pasantía de Investigación}{2016--2016}{Fundación IdeaPaís}{Santiago, Chile}{\item{Documento de investigación: “Ninis, no estudian ni trabajan”}}\detaileditem{Técnico en Administración de Empresas}{2009--2011}{DUOC UC}{Santiago, Chile}{\item{Técnico de nivel superior con mención en Marketing}}\detaileditem{Enseñanza Básica - Media}{1997--2008}{Colegio Marista Marcelino Champagnat}{Santiago, Chile}{\item{Educación Media Técnico Profesional}}}

\vspace{1 cm}

\begin{center}

\textbf{Nota:} Este documento fue elaborado en \href{https://rmarkdown.rstudio.com/}{\texttt{R-markdown}}. Para replicar el formato puede acceder al código en el siguiente \textbf{\href{https://github.com/JDConejeros/CV}{link}}.

\end{center}

\end{document}
